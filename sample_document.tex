\documentclass[12pt,a4paper]{article}
\usepackage[utf8]{inputenc}
\usepackage[T1]{fontenc}
\usepackage{amsmath,amsfonts,amssymb}
\usepackage{geometry}
\usepackage{graphicx}
\usepackage{float}
\usepackage{booktabs}
\usepackage{hyperref}
\usepackage{xcolor}
\usepackage{listings}
\usepackage{enumerate}
\usepackage{fancyhdr}

% Page setup
\geometry{left=2.5cm,right=2.5cm,top=2.5cm,bottom=2.5cm}
\pagestyle{fancy}
\fancyhf{}
\fancyhead[L]{Sample LaTeX Document}
\fancyhead[R]{\thepage}
\renewcommand{\headrulewidth}{0.4pt}

% Code listing style
\lstset{
    language=Python,
    basicstyle=\ttfamily\small,
    keywordstyle=\color{blue},
    commentstyle=\color{green},
    stringstyle=\color{red},
    numbers=left,
    numberstyle=\tiny,
    stepnumber=1,
    numbersep=5pt,
    backgroundcolor=\color{gray!10},
    frame=single,
    breaklines=true,
    captionpos=b
}

\title{\textbf{Sample LaTeX Document}\\
\large A Comprehensive Example of LaTeX Features}
\author{Your Name}
\date{\today}

\begin{document}

\maketitle

\tableofcontents
\newpage

\section{Introduction}

This document serves as a comprehensive example of various LaTeX features and formatting options. LaTeX is a powerful document preparation system that excels in typesetting mathematical expressions, creating professional documents, and handling complex formatting requirements.

\subsection{Why LaTeX?}

LaTeX offers several advantages:
\begin{itemize}
    \item Professional typesetting
    \item Excellent mathematical notation support
    \item Cross-referencing capabilities
    \item Bibliography management
    \item Consistent formatting
\end{itemize}

\section{Mathematical Expressions}

\subsection{Inline Mathematics}

Here are some examples of inline mathematical expressions: The quadratic formula is $x = \frac{-b \pm \sqrt{b^2 - 4ac}}{2a}$, and Euler's identity states that $e^{i\pi} + 1 = 0$.

\subsection{Display Mathematics}

Here's a more complex mathematical expression:

\begin{align}
    \int_{-\infty}^{\infty} e^{-x^2} dx &= \sqrt{\pi} \label{eq:gaussian}\\
    \frac{\partial^2 u}{\partial t^2} &= c^2 \nabla^2 u \label{eq:wave}\\
    \sum_{n=1}^{\infty} \frac{1}{n^2} &= \frac{\pi^2}{6} \label{eq:basel}
\end{align}

\subsection{Matrix Notation}

The following is an example of matrix notation:

\[
A = \begin{pmatrix}
    a_{11} & a_{12} & a_{13} \\
    a_{21} & a_{22} & a_{23} \\
    a_{31} & a_{32} & a_{33}
\end{pmatrix}
\]

\section{Tables}

\subsection{Simple Table}

\begin{table}[H]
\centering
\caption{Student Grades}
\begin{tabular}{lccc}
\toprule
Student & Math & Physics & Chemistry \\
\midrule
Alice & 95 & 87 & 92 \\
Bob & 78 & 85 & 88 \\
Carol & 92 & 91 & 89 \\
David & 85 & 79 & 93 \\
\bottomrule
\end{tabular}
\label{tab:grades}
\end{table}

\subsection{Advanced Table with Math}

\begin{table}[H]
\centering
\caption{Statistical Summary}
\begin{tabular}{lcc}
\toprule
\textbf{Variable} & \textbf{Mean} & \textbf{Standard Deviation} \\
\midrule
Height (cm) & $\mu = 170.5$ & $\sigma = 8.3$ \\
Weight (kg) & $\mu = 68.2$ & $\sigma = 12.1$ \\
Age (years) & $\mu = 25.7$ & $\sigma = 4.2$ \\
\bottomrule
\end{tabular}
\label{tab:stats}
\end{table}

\section{Code Listings}

\subsection{Python Code Example}

Here's an example of a Python function:

\begin{lstlisting}[caption=Fibonacci Function, label=lst:fibonacci]
def fibonacci(n):
    """
    Calculate the nth Fibonacci number using recursion.
    
    Args:
        n (int): The position in the Fibonacci sequence
        
    Returns:
        int: The nth Fibonacci number
    """
    if n <= 1:
        return n
    else:
        return fibonacci(n-1) + fibonacci(n-2)

# Example usage
for i in range(10):
    print(f"F({i}) = {fibonacci(i)}")
\end{lstlisting}

\section{Enumerated Lists}

\subsection{Numbered List}

The steps to solve a quadratic equation are:

\begin{enumerate}
    \item Identify the coefficients $a$, $b$, and $c$ in the equation $ax^2 + bx + c = 0$
    \item Calculate the discriminant: $D = b^2 - 4ac$
    \item Determine the nature of roots:
    \begin{itemize}
        \item If $D > 0$: Two distinct real roots
        \item If $D = 0$: One repeated real root
        \item If $D < 0$: Two complex conjugate roots
    \end{itemize}
    \item Apply the quadratic formula: $x = \frac{-b \pm \sqrt{D}}{2a}$
\end{enumerate}

\section{Theorems and Proofs}

\begin{theorem}[Pythagorean Theorem]
In a right-angled triangle, the square of the hypotenuse is equal to the sum of squares of the other two sides.
\end{theorem}

\begin{proof}
Let $a$ and $b$ be the lengths of the legs of a right triangle, and $c$ be the length of the hypotenuse. The area of the square with side length $a + b$ is $(a + b)^2$. This area can also be expressed as the sum of four triangles and the square of the hypotenuse: $4 \cdot \frac{1}{2}ab + c^2$.

Therefore:
\begin{align}
(a + b)^2 &= 4 \cdot \frac{1}{2}ab + c^2 \\
a^2 + 2ab + b^2 &= 2ab + c^2 \\
a^2 + b^2 &= c^2
\end{align}
\end{proof}

\section{References and Cross-references}

As we can see from equation \ref{eq:gaussian}, the Gaussian integral has a beautiful closed-form solution. Table \ref{tab:grades} shows the student performance data, and the Python code in listing \ref{lst:fibonacci} demonstrates a recursive implementation.

\section{Hyperlinks and URLs}

You can visit the \href{https://www.latex-project.org/}{official LaTeX website} for more information. For LaTeX documentation, check out \href{https://www.overleaf.com/learn}{Overleaf's learning resources}.

\section{Conclusion}

This document demonstrates various LaTeX features including:

\begin{itemize}
    \item Mathematical expressions and equations
    \item Tables with professional formatting
    \item Code listings with syntax highlighting
    \item Cross-referencing and labels
    \item Hyperlinks
    \item Theorems and proofs
    \item Lists and enumerations
\end{itemize}

LaTeX provides a powerful and flexible system for creating professional documents across various fields, from mathematics and science to humanities and business.

\end{document}

